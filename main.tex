\documentclass[a4paper, twocolumn]{article}
\usepackage[margin=1in]{geometry}
\usepackage[Glenn]{fncychap}
\usepackage{listings}
\usepackage{color}
\usepackage{verbatim}
\usepackage{geometry}
\usepackage{fancyhdr}

\title{Code Template for ACM-ICPC}
\author{kesarPista}

\definecolor{dkgreen}{rgb}{0,0.6,0}
\definecolor{gray}{rgb}{0.5,0.5,0.5}
\definecolor{mauve}{rgb}{0.58,0,0.82}
\setlength\columnseprule{0.4pt}
\geometry{a4paper,left=1cm,right=1cm,top=2cm,bottom=1.5cm}

\lstset{frame=tb,
  language=c++,
  aboveskip=3mm,
  belowskip=3mm,
  showstringspaces=false,
  columns=flexible,
  basicstyle={\small\ttfamily},
  numbers=none,
  numberstyle=\tiny\color{gray},
  keywordstyle=\color{blue},
  commentstyle=\color{dkgreen},
  stringstyle=\color{mauve},
  breaklines=true,
  breakatwhitespace=true
  tabsize=2
}

\begin{document}
\begin{titlepage}
\maketitle
\thispagestyle{empty}
\pagebreak
%\pagestyle{plain}
\pagestyle{fancy}
\lhead{}
\rhead{}
\chead{Code Template for ACM-ICPC, kesarPista}
\cfoot{}
\tableofcontents
\end{titlepage}

\pagestyle{fancy}
\chead{Code Template for ACM-ICPC, kesarPista}
\cfoot{- \thepage \ -}
  
\section{Template}
\subsection{Starter Code}
\lstinputlisting{codes/template_base.cpp}
\subsection{Script}
\lstinputlisting{codes/template_cprun.sh}
\subsection{Vim}
\lstinputlisting{codes/.template_vimrc}

\section{Number Theory}
\subsection{Chinese Remainder Theorem}
\lstinputlisting{codes/NT_crt.cpp}
\subsection{Extended Euclidean Algorithm}
\lstinputlisting{codes/NT_egcd.cpp}
\subsection{Mod Inverse}
\lstinputlisting{codes/NT_minv.cpp}
\subsection{Phi Function}
\lstinputlisting{codes/NT_phi.cpp}
\subsection{Linear Diophantine Equations}
\lstinputlisting{codes/NT_lde.cpp}
\subsection{Sieve}
\lstinputlisting{codes/NT_spf.cpp}


\section{Range-Query}
\subsection{Segment Tree}
\lstinputlisting{codes/RQ_segtree.cpp}
\subsection{Lazy Segement Tree}
\lstinputlisting{codes/RQ_lazysegtree.cpp}
\subsection{Fenwick Tree}
\lstinputlisting{codes/RQ_bit.cpp}
\subsection{Sparse Table}
\lstinputlisting{codes/RQ_rmq.cpp}


\section{String}
\subsection{Hashing}
\lstinputlisting{codes/S_hashing.cpp}
\subsection{String Trie}
\lstinputlisting{codes/S_trie.cpp}
\subsection{XOR Trie}
\lstinputlisting{codes/S_xortrie.cpp}


\section{Graph}
\subsection{DSU}
\lstinputlisting{codes/graph_dsu.cpp}
\subsection{Bridges}
\lstinputlisting{codes/graph_bridges.cpp}
\subsection{Bridges Online}
\lstinputlisting{codes/graph_bridges_online.cpp}
\subsection{Articulation Points}
\lstinputlisting{codes/graph_articulation.cpp}
\subsection{Kosaraju(SCC)}
\lstinputlisting{codes/graph_scc.cpp}
\subsection{Binary Lifting}
\lstinputlisting{codes/graph_binlift.cpp}


\section{Miscellaneous}
\subsection{Matrix Expo}
\lstinputlisting{codes/M_mat.cpp}
\subsection{Ordered Set}
\lstinputlisting{codes/M_ordset.cpp}
\subsection{NTT}
\lstinputlisting{codes/M_ntt.cpp}


\end{document}
